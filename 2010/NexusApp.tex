\documentclass{beamer}
\usepackage{beamerthemesplit}
\usepackage{graphics}
\logo{\includegraphics[height=1cm]{psi_logo_white.png}}
\usetheme{Pittsburgh}
\usecolortheme{dove}
\beamertemplatenavigationsymbolsempty
\setbeamertemplate{footline}[frame number]
\definecolor{myback}{RGB}{175,238,238}
\setbeamercolor{structure}{bg=myback}
\usepackage[T1]{fontenc}
\newcommand{\changefont}[3] {
 \fontfamily{#1} \fontseries{#2} \fontshape{#3} \selectfont}


\begin{document}

\begin{frame}
\frametitle{NeXus Application Definitions}
\begin{itemize}
\item<1->Application definitions are the NeXus way of defining a standard
\item<2->Describe the minimum data required for typical usage 
\item<3->Strive to cover 80-90\% of all use cases
\item<4->Expressed in NXDL
\item<5->I will show example files and ASCII metadata dumps
\item<6->{\color{blue}A real file will always contain more data and this will not break the standard compliance!} 
\end{itemize}
\end{frame}


\begin{frame}
\frametitle{Scans and Rastering}
\begin{itemize}
\item Come in all shapes and sizes
\item Captured by rules:
\begin{itemize}
\item Store all varied parameters as arrays of length NP at the appropriate place in the NeXus 
 hierarchy
\item For multi detectors, NP, number of scan points is always the first dimension
\item In NXdata: create links to counts and varied variables
\end{itemize}
\end{itemize}
\end{frame}


\begin{frame}
\frametitle{Scan Example 1: rotating sample}

\begin{tabbing}
\hspace*{1cm} \= \hspace*{1cm} \= \hspace*{1cm} \= \hspace*{1cm} \= \hspace*{1cm} \= \hspace*{1cm}\= \kill
entry:NXentry \\
 \>sample:NXsample\\
 \> \> rotation\_angle[NP], axis=1 (1) \\
 \> instrument:NXinstrument\\
 \>  \>detector:NXdetector\\
 \>  \> \>data[NP],signal=1 (2)\\
 \>control:NXmonitor\\  
 \> \>data[NP]\\  
 \>data:NXdata\\
 \> \> link to (1)\\
 \> \> link to (2) \\
\end{tabbing}
\end{frame}

\begin{frame}
\frametitle{Scan Example 2: complex scan in Q}

\begin{tabbing}
\hspace*{1cm} \= \hspace*{1cm} \= \hspace*{1cm} \= \hspace*{1cm} \= \hspace*{1cm} \= \hspace*{1cm}\= \kill
entry:NXentry \\
 \>sample:NXsample\\
 \> \> rotation\_angle[NP], axis=1 (1) \\
 \> \> phi[NP], axis=1 (2) \\
 \> \> chi[NP], axis=1 (3) \\
 \> \> h[NP], axis=1 (4), primary=1 \\
 \> \> k[NP], axis=1 (5) \\
 \> \> l[NP], axis=1 (6) \\
 \> instrument:NXinstrument\\
 \>  \>detector:NXdetector\\
 \>  \> \>data[NP],signal=1 (7)\\
 \>  \> \>polar\_angle[NP],signal=1 (8)\\
 \>data:NXdata\\
 \> \> link to (1)\\
 \> \> link to (2) \\
 \> \> link to (...) \\
 \> \> link to (8) \\
\end{tabbing}
\end{frame}


\begin{frame}
\frametitle{Scan Example 3: sample rotation, area detector}
\begin{tabbing}
\hspace*{1cm} \= \hspace*{1cm} \= \hspace*{1cm} \= \hspace*{1cm} \= \hspace*{1cm} \= \hspace*{1cm}\= \kill
entry:NXentry \\
 \>sample:NXsample\\
 \> \> rotation\_angle[NP], axis=1 (1) \\
 \> instrument:NXinstrument\\
 \>  \>detector:NXdetector\\
 \>  \> \>data[NP,xsize,ysize],signal=1 (2)\\
 \>control:NXmonitor\\  
 \> \>data[NP]\\  
 \>data:NXdata\\
 \> \> link to (1)\\
 \> \> link to (2) \\
\end{tabbing}
\end{frame}


\end{document}

